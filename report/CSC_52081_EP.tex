\documentclass[journal, a4paper]{IEEEtran}

\usepackage{graphicx}
\usepackage{url}
\usepackage{amssymb}
\usepackage{amsmath}
\usepackage{subfiles}

% Some useful/example abbreviations for writing math
\newcommand{\argmax}{\operatornamewithlimits{argmax}}
\newcommand{\argmin}{\operatornamewithlimits{argmin}}
\newcommand{\x}{\mathbf{x}}
\newcommand{\y}{\mathbf{y}}
\newcommand{\ypred}{\mathbf{\hat y}}
\newcommand{\yp}{{\hat y}}

\begin{document}

% Define document title, do NOT write author names for the initial submission
\title{Reinforcement Learning Project}
\author{Anonymous Authors (recall: the entire document should be anonymous)}
\maketitle

% Write abstract here
\begin{abstract}
	This is a rough guide to producing the project report. 
	The structure outlined here is a suggestion, but you must use this template\footnote{There are also Word templates for this format (\url{https://www.ieee.org/conferences/publishing/templates.html}) if you wish; but don't forget you will submit a pdf} (of course, replace these hints/instructions/examples with your own text); a limit of 5 pages \emph{not including} references and an optional appendix. 
	%Hint: shared tools like \texttt{http://overleaf.com/} are great for collaborating on a multi-author report in \LaTeX. 
\end{abstract}

% Each section begins with a \section{title} command
\subfile{sections/CSC_52081_EP_I.tex}
\subfile{sections/CSC_52081_EP_II.tex}
\subfile{sections/CSC_52081_EP_III.tex}
\subfile{sections/CSC_52081_EP_IV.tex}
\subfile{sections/CSC_52081_EP_V.tex}

\newpage
\section*{Appendix}
This is the place to put work that you did but is not essential to understand the main outcome of your work. For example, additional results and tables, lengthy proofs or derivations. Material here does not count towards page limit (but also it will be optional for the reviewer/teacher to work through). 
\end{document}