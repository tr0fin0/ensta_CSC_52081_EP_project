\documentclass[../CSC_52081_EP.tex]{subfiles}

\begin{document}
    \section{Background ($\approx 1$ page)}
    \label{sec:background}

    Here, tell the reader everything they need to know to understand your work, in your own words. The main emphasis in this section: be pedagogical, target your reader as someone who has followed the course,  and needs to be reminded of relevant concepts to understand what you have done. 

    You must properly credit any source (textbook, article, course notes/slides) via a suitable reference,  e.g., \cite{RLBook}, Chapter 3, or \cite{Lecture4}, or even a blog post you found on the web \cite{Post3}. 

    Here is the place to introduce your notation, (e.g., state $s$, policy $\pi_\theta$ parametrized by $\theta$, trajectory $\boldsymbol\tau \sim p_\theta$ from policy $\pi_\theta$ under environment $p$), but make sure each part of your notation is stated somewhere. 

\end{document}
