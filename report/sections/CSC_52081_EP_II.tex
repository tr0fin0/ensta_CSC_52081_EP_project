\documentclass[../CSC_52081_EP.tex]{subfiles}

\begin{document}
    \section{Background}
    \label{sec:background}

    % Here, tell the reader everything they need to know to understand your work, in your own words. The main emphasis in this section: be pedagogical, target your reader as someone who has followed the course,  and needs to be reminded of relevant concepts to understand what you have done. 

    % You must properly credit any source (textbook, article, course notes/slides) via a suitable reference,  e.g., \cite{RLBook}, Chapter 3, or \cite{Lecture4}, or even a blog post you found on the web \cite{Post3}. 

    % Here is the place to introduce your notation, (e.g., state $s$, policy $\pi_\theta$ parametrized by $\theta$, trajectory $\boldsymbol\tau \sim p_\theta$ from policy $\pi_\theta$ under environment $p$), but make sure each part of your notation is stated somewhere. 

    \subsection{Discrete Action Space}

    DEEP Q-Network (DQN) and SARSA are powerful reiforcement learning algorithms to solve discrete action control problems, suitable to our first approach to the Car Racing environment.
    Both approaches are based on the Q-learning algorithm, which is a model-free reinforcement learning algorithm that aims to learn the optimal action-value function $Q(s, a)$, where $s$ is the state and $a$ is the action.
    However, DQN uses a deep neural network to approximate the Q-function, while SARSA uses a table to store the Q-values \cite{Popular_RL}. To account for the limitations of the SARSA approach in limited high-dimentional state spaces (as the Car Racing environment with an observation space of 96x96x3), we will explore a modern approach called Deep SARSA. \cite{Deep_SARSA}

    Deep SARSA combines the on-policy nature of traditional SARSA with neural network architectures to handle large state spaces effectively. Unlike DQN's off-policy approach, Deep SARSA maintains SARSA's fundamental characteristics while scaling to complex environments.

    On-policy learning methods, such as SARSA, updates the Q-values using actions selected by the current policy. The usual update rule follows the equation:
    \begin{equation}
        Q(s, a) \leftarrow Q(s, a) + \alpha \left[ r + \gamma Q(s', \pi(s')) - Q(s, a) \right]
    \end{equation}

    where $\alpha$ is the learning rate, $r$ is the reward, $\gamma$ is the discount factor, and $\pi(s')$ is the action selected by the policy at the next state $s'$ (usually the $\epsilon$-greedy policy).
    The $\epsilon$-greedy policy selects a random action (explore) with probability $\epsilon$ and the best action (exploit) with probability 1-$\epsilon$.
    
    On the other hand, off-policy learning methods, such as DQN, updates the Q-values using actions selected by a different policy. The update rule is given by:
    \begin{equation}
        Q(s, a) \leftarrow Q(s, a) + \alpha \left[ r + \gamma \max_{a'} Q(s', a') - Q(s, a) \right]
    \end{equation}
    
    where $\max_{a'} Q(s', a')$ is the maximum Q-value at the next state $s'$ (greedy selection).

    On-policy methods tend to be more stable, but they can be less sample-efficient and must actively explore during training.In contrast, off-policy methods can learn from past experiences, which can lead to better exploration and exploitation of the environment, but may require careful tuning to ensure stability.

    \subsection{Continuous Action Space}

    For the second approach to the Car Racing environment, we will explore algorithms that can handle continuous action spaces.
    Starting from an evolutionary approach, we will test the performance of the Cross-Entropy Method (CEM) and the (1+1) Successive Adaptation Evolution Strategy (SA-ES) in the Car Racing environment.
    Both algorithms are based on the idea of optimizing a distribution of policies to maximize the expected return in the environment.

    The Cross-Entropy Method is a simple optimization algorithm that iteratively samples policies from a Gaussian distribution and updates the distribution parameters to maximize the expected return.
    It generates multiple candidate solutions (policies), ranks them based on performance, and updates the policy distribution using the best-performing candidates. It is efficient in high-dimentional spaces and can discover complex plocies through population diversity. It follows the steps:
    \begin{enumerate}
        \item Sample $N$ policies from a Gaussian distribution.
        \item Evaluate the policies in the environment.
        \item Select the top $M$ policies.
        \item Update the distribution parameters to fit the selected policies.
        \item Repeat until convergence.
    \end{enumerate}

    The (1+1) Successive Adaptation Evolution Strategy is a variant of the Evolution Strategy algorithm that uses a single parent policy to generate a single child policy at each iteration.
    It is a simple and efficient optimization algorithm that can handle continuous action spaces effectively. It uses a Gaussian mutation operator to perturb the parent policy and keeps the new version if it performs better in the environment.
    The amount of change (mutation size) is adjusted dynamically to balance exploration and exploitation, making it suitable for continuous updates, and more sample-efficient than other evolutionary algorithms. It follows the steps:
    \begin{enumerate}
        \item Sample a policy from a Gaussian distribution.
        \item Evaluate the policy in the environment.
        \item Update the parent policy if the child policy performs better.
        \item Adjust the mutation size based on the performance.
        \item Repeat until convergence.
    \end{enumerate}
    

    Secondly, we will explore and compare two policy-based methods designed for continuous control: Proximal Policy Optimization (PPO) \cite{PPO} and Soft Actor-Critic (SAC) \cite{SAC} in the Car Racing environment.
    PPO is an on-policy gradient method, meaning it directly updates the policy (instead of learning a value function like Q-learning). It uses an actor-critic architecture to learn a parameterized policy that maximizes the expected return. It follows the steps:
    \begin{enumerate}
        \item Collect trajectories using the current policy.
        \item Compute the advantage function using the critic network.
        \item Update the policy using the advantage function and the policy gradient.
        \item Repeat until convergence.
    \end{enumerate}

    SAC is an off-policy (meaning it reuses past experiences for learning) actor-critic method that uses the maximum entropy framework to encourage exploration and improve sample efficiency. It learns a stochastic policy that maximizes the expected return while maximizing the entropy of the policy (avoiding premature convergence to suboptimal policies). It follows the steps:
    \begin{enumerate}
        \item Collect trajectories using the current policy.
        \item Compute the advantage function using the critic network.
        \item Update the policy using the advantage function and the policy gradient.
        \item Update the critic network using the temporal difference error.
        \item Repeat until convergence.
    \end{enumerate}

    \subsection{Objective}
    In this study, we will compare the performance of Deep SARSA and DQN in the Car Racing environment to understand the trade-offs between on-policy and off-policy learning methods in the context of discrete action space. Conversely, we also aim to compare the performance of CEM, SA-ES, PPO, and SAC to understand the trade-offs between evolutionary algorithms and policy-based methods in continuous action spaces.

\end{document}
