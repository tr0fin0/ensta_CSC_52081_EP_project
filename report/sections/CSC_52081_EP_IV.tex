\documentclass[../CSC_52081_EP.tex]{subfiles}

\begin{document}
    \section{Results and Discussion ($\approx 1$-2 pages)}

    Show the results (tables, plots, \ldots), and -- most importantly -- discuss and \emph{interpret} the results. Do not just narrate what you did and report results in the tables, but also \emph{discuss the implications of the results}. Method A beats method B; but: Why? How? In which contexts?

    Negative results are also results. Your agent performed poorly or not as expected? If you can explain why then this is indeed an important result. 

    Always discuss limitations, whether observed in your results or suspected in different scenarios. 

    Make use of plots, e.g., Fig.~\ref{results_figure}, tables, etc.; anything that illustrates the performance of your agent in the environment under different configurations. Make sure to clearly indicate the parametrization behind each result (e.g., $\gamma$, or whatever is relevant to your experiments). 

    % \begin{figure}[!ht]
    % 	\centering
    % 	\includegraphics[width=0.7\columnwidth]{fig/results.pdf}
    % 	\caption{\label{results_figure}Your plots should be as self-explanatory (from the legend, axis labels, and the caption) as possible.}
    % \end{figure}

    %\begin{table}[!ht]
    %	\caption{\label{results_table}Table captions should adequately describe the contents of tables (unlike this one).}
    %	\centering
    %	\begin{tabular}{lll}
    %		\hline
    %		\textbf{Environment config.} & \textbf{SARSA} & \textbf{Q-Learning}  \\
    %		\hline
    %		Simulation 1        & 10             & 15 \\
    %		Simulation 2        & 12             & 11 \\
    %		\hline
    %	\end{tabular}
    %\end{table}
\end{document}
